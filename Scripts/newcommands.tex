
\pgfdeclarelayer{ultrabackground}
\pgfdeclarelayer{background}
\pgfdeclarelayer{foreground}
\pgfsetlayers{ultrabackground,background,main,foreground}


\newcommand{\JustPrimary}		[1]{\textcolor{\StrongPrimary}{#1}}
\newcommand{\ItPrimary}			[1]{\textcolor{\StrongPrimary}{\textit{#1}}}
\newcommand{\BoldPrimary}		[1]{\textcolor{\StrongPrimary}{\textbf{#1}}}
\newcommand{\BoldItPrimary}		[1]{\textcolor{\StrongPrimary}{\textit{ \textbf{#1} }}}
\newcommand{\ItBoldPrimary}		[1]{\BoldItPrimary{#1}}
%
\newcommand{\JustSecondary}		[1]{\textcolor{\StrongSecondary}{#1}}
\newcommand{\ItSecondary}		[1]{\textcolor{\StrongSecondary}{\textit{#1}}}
\newcommand{\BoldSecondary}		[1]{\textcolor{\StrongSecondary}{\textbf{#1}}}
\newcommand{\BoldItSecondary}	[1]{\textcolor{\StrongSecondary}{\textit{ \textbf{#1} }}}
\newcommand{\ItBoldSecondary}	[1]{\BoldItSecondary{#1}}



\newcommand {\PrimaryRectangle} [1]
{
	\begin{center}
	\begin{tikzpicture}
		%
		\node
		[
			shape			= rectangle,		% shape
			rounded corners	= 0.2cm,			% shape
			minimum width	= 0.7cm,			%
			minimum height	= 0.7cm,			%
			line width		= 0cm,				% thickness of the border
			fill			= \SoftPrimary,		%
			draw			= \StrongPrimary,	% draw the border with this color
			line width		= 0.1cm,			% thickness
			text width		= 0.8\textwidth,	% max. width of the text
			align			= center,			% text alignment
			inner xsep		= 0.2cm,			%
			inner ysep		= 0.2cm,			%
		]
		{#1};
		%
	\end{tikzpicture}
	\end{center}
}


\newcommand {\SecondaryRectangle} [1]
{
	\begin{center}
	\begin{tikzpicture}
		%
		\node
		[
			shape			= rectangle,		% shape
			rounded corners	= 0.2cm,			% shape
			minimum width	= 0.7cm,			%
			minimum height	= 0.7cm,			%
			line width		= 0cm,				% thickness of the border
			fill			= \SoftSecondary,	%
			draw			= \StrongSecondary,	% draw the border with this color
			line width		= 0.1cm,			% thickness
			text width		= 0.8\textwidth,	% max. width of the text
			align			= center,			% text alignment
			inner xsep		= 0.2cm,			%
			inner ysep		= 0.2cm,			%
		]
		{#1};
		%
	\end{tikzpicture}
	\end{center}
}


\newcommand {\PrimaryRectangleWithCaption} [3]
{
	\begin{center}
	\begin{tikzpicture}
		%
		\node (a)
		[
			shape			= rectangle,		% shape
			rounded corners	= 0.2cm,			% shape
			minimum width	= 0.7cm,			%
			minimum height	= 0.7cm,			%
			fill			= \SoftPrimary,		%
			draw			= \StrongPrimary,	% draw the border with this color
			line width		= 0.1cm,			% thickness
			text width		= 0.8\textwidth,	% max. width of the text
			align			= center,			% text alignment
			inner xsep		= 0.3cm,			%
			inner ysep		= 0.3cm,			%
		]
		{#2};
		%
		\node
		[
			shape			= rectangle,		% shape
			rounded corners	= 0.2cm,			% shape
			anchor			= mid,
			fill			= \SoftPrimary,		%
			draw			= \StrongPrimary,	% draw the border with this color
			text			= \StrongPrimary,	%
			align			= center,			% text alignment
			line width		= 0.1cm,			% thickness
			inner xsep		= 0.2cm,			%
			inner ysep		= 0.2cm,			%
		]
		at (a.#3)
		{#1};
		%
	\end{tikzpicture}
	\end{center}
}


\newcommand {\SecondaryRectangleWithCaption} [3]
{
	\begin{center}
	\begin{tikzpicture}
		%
		\node (a)
		[
			shape			= rectangle,		% shape
			rounded corners	= 0.2cm,			% shape
			minimum width	= 0.7cm,			%
			minimum height	= 0.7cm,			%
			fill			= \SoftSecondary,	%
			draw			= \StrongSecondary,	% draw the border with this color
			line width		= 0.1cm,			% thickness
			text width		= 0.8\textwidth,	% max. width of the text
			align			= center,			% text alignment
			inner xsep		= 0.3cm,			%
			inner ysep		= 0.3cm,			%
		]
		{#2};
		%
		\node
		[
			shape			= rectangle,		% shape
			rounded corners	= 0.2cm,			% shape
			anchor			= mid,
			fill			= \SoftSecondary,	%
			draw			= \StrongSecondary,	% draw the border with this color
			text			= \StrongSecondary,	%
			align			= center,			% text alignment
			line width		= 0.1cm,			% thickness
			inner xsep		= 0.2cm,			%
			inner ysep		= 0.2cm,			%
		]
		at (a.#3)
		{#1};
		%
	\end{tikzpicture}
	\end{center}
}




% ~~~~~~~~~~~~~~~~~~~~~~~~~~~~~~~~~~~~~~~~~~~~~~~~~~~~~~~~~~~~~ %
\newcommand{\EstimatedSignal}						[0]	{\widehat{\Signal}}
\newcommand{\OptimallyEstimatedSignal}				[0]	{\widehat{\Signal}^{*}}
\newcommand{\OptimallyBayesEstimatedSignal}			[0]	{\widehat{\Signal}^{*}_{\text{Bayes}}}
\newcommand{\OptimallyCostFunctionEstimatedSignal}	[0]	{\widehat{\Signal}^{*}_{\text{c.f.}}}
%
\newcommand{\EstimatedSignalAt}						[1]	{\EstimatedSignal \left( #1 \right)}
\newcommand{\OptimallyEstimatedSignalAt}			[1]	{\OptimallyEstimatedSignal \left( #1 \right)}
\newcommand{\OptimallyBayesEstimatedSignalAt}		[1]	{\OptimallyBayesEstimatedSignal \left( #1 \right)}
\newcommand{\OptimallyCostFunctionEstimatedSignalAt}[1]	{\OptimallyCostFunctionEstimatedSignal \left( #1 \right)}
%
\newcommand{\EstimatedSignalAtSet}						[1]	{\EstimatedSignal_{#1}}
\newcommand{\OptimallyEstimatedSignalAtSet}				[1]	{\OptimallyEstimatedSignal_{#1}}
\newcommand{\OptimallyBayesEstimatedSignalAtSet}		[1]	{\OptimallyBayesEstimatedSignal_{#1}}
\newcommand{\OptimallyCostFunctionEstimatedSignalAtSet}	[1]	{\OptimallyCostFunctionEstimatedSignal_{#1}}




\newcommand{\EstimatedEigenfunctionWeight}			[1]
			{\widehat{a}_{#1}}
\newcommand{\SetOfEstimatedEigenfunctionsWeights}	[0]
			{\widehat{\SetOfEigenfunctionsWeights}}
\newcommand{\EigenfunctionsWeightsEstimationError}	[0]
			{\widetilde{\SetOfEigenfunctionsWeights}}




\newcommand{\EstimatedEigenfunctionWeightOfSensor}			[2]
			{\widehat{a}_{#1, #2}}
\newcommand{\SetOfEstimatedEigenfunctionsWeightsOfSensor}	[1]
			{\widehat{\SetOfEigenfunctionsWeights}_{#1}}
\newcommand{\EigenfunctionsWeightsEstimationErrorOfSensor}	[1]
			{\widetilde{\SetOfEigenfunctionsWeights}_{#1}}
\newcommand{\LocallyEstimatedEigenfunctionWeightOfSensor}	[2]
			{\widehat{a}_{#1, #2}^{loc}}




\newcommand{\SetOfLocallyEstimatedEigenfunctionsWeights}				[0]
			{\widehat{\SetOfEigenfunctionsWeights}^{\text{loc}}}
\newcommand{\SetOfLocallyEstimatedEigenfunctionsWeightsOfSensor}		[1]
			{\widehat{\SetOfEigenfunctionsWeights}^{\text{loc}}_{#1}}
\newcommand{\SetOfLocallyBayesEstimatedEigenfunctionsWeights}			[0]
			{\widehat{\SetOfEigenfunctionsWeights}_{\text{Bayes}}^{\text{loc}}}
\newcommand{\SetOfLocallyCostFunctionEstimatedEigenfunctionsWeights}	[0]
			{\widehat{\SetOfEigenfunctionsWeights}_{\text{c.f.}}^{\text{loc}}}
%
\newcommand{\SetOfLocallyBayesEstimatedEigenfunctionsWeightsOfSensor}			[1]
			{\widehat{\SetOfEigenfunctionsWeights}_{\text{Bayes}, #1}^{\text{loc}}}
\newcommand{\SetOfLocallyCostFunctionEstimatedEigenfunctionsWeightsOfSensor}	[1]
			{\widehat{\SetOfEigenfunctionsWeights}_{\text{c.f.}, #1}^{\text{loc}}}
%
%
\newcommand{\SetOfCentrallyEstimatedEigenfunctionsWeights}				[0]
			{\widehat{\SetOfEigenfunctionsWeights}^{\text{cen}}}
\newcommand{\SetOfCentrallyBayesEstimatedEigenfunctionsWeights}			[0]
			{\widehat{\SetOfEigenfunctionsWeights}_{\text{Bayes}}^{\text{cen}}}
\newcommand{\SetOfCentrallyCostFunctionEstimatedEigenfunctionsWeights}	[0]
			{\widehat{\SetOfEigenfunctionsWeights}_{\text{c.f.}}^{\text{cen}}}
%
\newcommand{\SetOfCentrallyBayesEstimatedEigenfunctionsWeightsOfSensor}			[1]
			{\widehat{\SetOfEigenfunctionsWeights}_{\text{Bayes}, #1}^{\text{cen}}}
\newcommand{\SetOfCentrallyCostFunctionEstimatedEigenfunctionsWeightsOfSensor}	[1]
			{\widehat{\SetOfEigenfunctionsWeights}_{\text{c.f.}, #1}^{\text{cen}}}
%
%
\newcommand{\SetOfDistributelyEstimatedEigenfunctionsWeights}				[0]
			{\widehat{\SetOfEigenfunctionsWeights}^{\text{dis}}}
\newcommand{\SetOfDistributelyBayesEstimatedEigenfunctionsWeights}			[0]
			{\widehat{\SetOfEigenfunctionsWeights}_{\text{Bayes}}^{\text{dis}}}
\newcommand{\SetOfDistributelyCostFunctionEstimatedEigenfunctionsWeights}	[0]
			{\widehat{\SetOfEigenfunctionsWeights}_{\text{c.f.}}^{\text{dis}}}
%
\newcommand{\SetOfDistributelyBayesEstimatedEigenfunctionsWeightsOfSensor}			[1]
			{\widehat{\SetOfEigenfunctionsWeights}_{\text{Bayes}, #1}^{\text{dis}}}
\newcommand{\SetOfDistributelyCostFunctionEstimatedEigenfunctionsWeightsOfSensor}	[1]
			{\widehat{\SetOfEigenfunctionsWeights}_{\text{c.f.}, #1}^{\text{dis}}}




\newcommand{\MaxAdmittableDistanceBtwOptimalCentralizedAndSuboptimalDistributedEstimates}	[0]
			{d_{\text{o.c.}-\text{s.d.}}^{\text{max}}}



\newcommand{\DistanceBtwLocalApproximatedBayesAndLocalApproximatedCostFunction}	[0]
	{d_{\text{l.a.B. - l.a.c.f.}}}
\newcommand{\DistanceBtwLocalOptimalBayesAndLocalApproximatedCostFunction}		[0]
	{d_{\text{l.o.B. - l.a.c.f.}}}
\newcommand{\DistanceBtwLocalOptimalBayesAndLocalApproximatedBayes}				[0]
	{d_{\text{l.o.B. - l.a.B.}}}



\newcommand{\NaiveDistributedEstimatorOfEigenfunctionsWeights}		[0]
	{\widehat{\SetOfEigenfunctionsWeights}_{\text{dis}}^{\text{naive}}}
%
\newcommand{\GuessedDistributedEstimatorOfEigenfunctionsWeights}	[0]
	{\widehat{\SetOfEigenfunctionsWeights}_{\text{dis}}^{\text{guess}}}
\newcommand{\GuessedDistributedEstimatorOfEigenfunctionsWeightsOfGuess}	[1]
	{\GuessedDistributedEstimatorOfEigenfunctionsWeights \left( #1 \right)}




\newcommand{\DefinedAs}			[0]	{\mathrel{\mathop:}=}
\newcommand{\IDefinedAs}		[0]	{=\mathrel{\mathop:}}


\newcommand{\ExponentialOf}		[1]	{\mathrm{exp} \left( #1 \right)}
\newcommand{\LogarithmOf}		[1]	{\mathrm{log} \left( #1 \right)}
\newcommand{\ConvexHullOf}		[1]	{\mathrm{c.h.} \left( #1 \right)}


\newcommand{\MaximumOfOne}		[1]	{\mathrm{max} \left\lbrace #1 \right\rbrace}
\newcommand{\MaximumOfTwo}		[2]	{\mathrm{max} \left\lbrace #1, #2 \right\rbrace)}
\newcommand{\MaximumOfThree}	[3]	{\mathrm{max} \left\lbrace #1, #2, #3 \right\rbrace)}
\newcommand{\MinimumOfOne}		[1]	{\mathrm{min} \left\lbrace #1 \right\rbrace}
\newcommand{\MinimumOfTwo}		[2]	{\mathrm{min} \left\lbrace #1, #2 \right\rbrace)}
\newcommand{\MinimumOfThree}	[3]	{\mathrm{min} \left\lbrace #1, #2, #3 \right\rbrace)}


\newcommand{\CostFunction}				[0]	{Q}
\newcommand{\CostFunctionOf}			[1]	{\CostFunction \left( #1 \right)}
\newcommand{\CostFunctionOfSensor}		[1]	{\CostFunction_{#1}}
\newcommand{\CostFunctionOfSensorOf}	[2]	{\CostFunctionOfSensor{#1} \left( #2 \right)}


\newcommand{\KroneckerDeltaOf}			[2]	{\delta_{#1 #2}}


\newcommand{\FloorOf}			[1]	{\lfloor #1 \rfloor}
\newcommand{\CeilOf}			[1]	{\lceil #1 \rceil}


\newcommand{\GammaFunctionOf}	[1]	{\Gamma \left( #1 \right)}

\newcommand{\SensorIndex}			{s}
\newcommand{\SensorIndexTwo}		{r}
\newcommand{\SensorIndexThree}		{q}
%
\newcommand{\MeasurementIndex}		{m}
\newcommand{\MeasurementIndexTwo}	{n}
\newcommand{\MeasurementIndexThree}	{p}
%
\newcommand{\TimeIndex}				{t}
\newcommand{\TimeIndexTwo}			{\tau}
\newcommand{\TimeIndexThree}		{\tau'}
%
\newcommand{\EigenvalueIndex}		{k}
\newcommand{\EigenvalueIndexTwo}	{i}
\newcommand{\EigenvalueIndexThree}	{g}
%
\newcommand{\EigenvectorIndex}		{\EigenvalueIndex}
\newcommand{\EigenvectorIndexTwo}	{\EigenvalueIndexTwo}
\newcommand{\EigenvectorIndexThree}	{\EigenvalueIndexThree}
%
\newcommand{\EigenfunctionIndex}		{\EigenvalueIndex}
\newcommand{\EigenfunctionIndexTwo}		{\EigenvalueIndexTwo}
\newcommand{\EigenfunctionIndexThree}	{\EigenvalueIndexThree}



\newcommand{\NumberOfEigenvalues}					[0]	{E}
\newcommand{\NumberOfEigenvectors}					[0]	{\NumberOfEigenvalues}
\newcommand{\NumberOfEigenfunctions}				[0]	{\NumberOfEigenvalues}
\newcommand{\NumberOfSensors}						[0]	{S}
\newcommand{\NumberOfMeasurements}					[0]	{M}
\newcommand{\NumberOfMeasurementsOfSensor}			[1]	{\NumberOfMeasurements_{#1}}
\newcommand{\NumberOfMeasurementsOfAuxiliaryGrid}	[0]	{\NumberOfMeasurements_{G}}
\newcommand{\NumberOfInputLocations}				[0]	{\NumberOfMeasurements}
\newcommand{\NumberOfInputLocationsOfSensor}		[1]	{\NumberOfMeasurements_{#1}}
\newcommand{\NumberOfInputLocationsOfAuxiliaryGrid}	[0]	{\NumberOfMeasurements_{G}}
\newcommand{\NumberOfTimeIndexes}					[0]	{T}



\newcommand{\DimensionOfInputLocationsDomain}		[0]	{D}
\newcommand{\IdentityMatrix}		[1]	{I_{#1}}
\newcommand{\OnesVector}			[1]	{\mathds{1}_{#1}}

\newcommand{\TraceOf}				[1]	{\text{tr} \left( #1 \right)}
\newcommand{\DeterminantOf}			[1]	{\text{det} \left( #1 \right)}
\newcommand{\SetOfEigenvaluesOf}	[1]	{\text{eig} \left( #1 \right)}

\newcommand{\Eigenvalue}			[1]	{\lambda_{#1}}
\newcommand{\Eigenvector}			[1]	{v_{#1}}
\newcommand{\MaximalEigenvalue}		[0]	{\lambda_{\text{max}}}
\newcommand{\MinimalEigenvalue}		[0]	{\lambda_{\text{min}}}

\newcommand{\SpectralRadius}		[0]	{\mu}
\newcommand{\SpectralRadiusOf}		[1]	{\SpectralRadius \left( #1 \right)}


\newcommand{\GaussianDistribution}					[2]	{\mathcal{N} \left( #1, #2 \right)}
\newcommand{\GammaDistribution}						[2]	{\text{Gamma} \left( #1, #2 \right)}
\newcommand{\ChiSquareDistribution}					[0]	{\chi^{2}}
\newcommand{\ChiSquareDistributionOfIndex}			[1]	{\ChiSquareDistribution \left( #1 \right)}
\newcommand{\InverseChiSquareDistribution}			[0]	{\text{Inv-}\chi^{2}}
\newcommand{\InverseChiSquareDistributionOfIndex}	[1]	{\InverseChiSquareDistribution \left( #1 \right)}
\newcommand{\UniformDistribution}					[2]	{\mathcal{U} \left[ #1, #2 \right]}
\newcommand{\ExponentialDistribution}				[1]	{\text{Exp} \left( #1 \right)}


\newcommand{\SetOfRealNumbers}						[0]	{\mathbb{R}}
\newcommand{\SetOfRealPositiveNumbers}				[0]	{\mathbb{R}_{+}}
\newcommand{\SetOfNaturalNumbers}					[0]	{\mathbb{N}}
\newcommand{\SetOfNaturalPositiveNumbers}			[0]	{\mathbb{N}_{+}}

\newcommand{\SetOfSquareSummableInfiniteVectors}			[0]	{\ell}
\newcommand{\SetOfWeightedSquareSummableInfiniteVectors}	[0]	{\textsl{l}_{\RKHS}}
\newcommand{\SetOfSquareIntegrableFunctions}				[0]	{\textsl{L}^{2}}
\newcommand{\SetOfSquareIntegrableFunctionsIn}				[1]	{\textsl{L}^{2} \left( #1 \right)}

\newcommand{\Signal}							[0]	{f}
\newcommand{\SignalOne}							[0]	{f}
\newcommand{\SignalTwo}							[0]	{g}
\newcommand{\SignalAt}							[1]	{\Signal \left( #1 \right)}
\newcommand{\SignalOneAt}						[1]	{\SignalOne \left( #1 \right)}
\newcommand{\SignalTwoAt}						[1]	{\SignalTwo \left( #1 \right)}



\newcommand{\StochasticProcess}					[0]	{\mathcal{F}}
\newcommand{\StochasticProcessRealization}		[0]	{\textsl{f}}
\newcommand{\StochasticProcessAt}				[1]	{\StochasticProcess \left( #1 \right)}
\newcommand{\StochasticProcessRealizationAt}	[1]	{\StochasticProcessRealization \left( #1 \right)}



\newcommand{\InfiniteVector}				[0]	{\mathbf{v}}
\newcommand{\InfiniteVectorOne}				[0]	{\mathbf{v}}
\newcommand{\InfiniteVectorTwo}				[0]	{\mathbf{w}}



\newcommand{\InputLocationsDomain}			[0] {\mathcal{D}}
\newcommand{\InputLocationsDomainOfSensor}	[1] {\InputLocationsDomain_{#1}}
\newcommand{\MeasurementsDomain}			[0] {\mathcal{M}}
\newcommand{\MeasurementsDomainOfSensor}	[1] {\MeasurementsDomain_{#1}}



\newcommand{\InputLocation}							[0]	{x}
\newcommand{\InputLocationOne}						[0]	{x}
\newcommand{\InputLocationTwo}						[0]	{x'}
\newcommand{\InputLocationOfSensor}					[1]	{\InputLocation_{#1}}
\newcommand{\InputLocationOfIndex}					[1]	{\InputLocation^{#1}}
\newcommand{\InputLocationOfSensorAndIndex}			[2]	{\InputLocationOfSensor{#1}^{#2}}
\newcommand{\InputLocationOfAuxiliaryGridAndIndex}	[1]	{\InputLocationOfSensor{G}^{#1}}
%
\newcommand{\SetOfInputLocations}					[0]	{\mathcal{X}}
\newcommand{\SetOfInputLocationsOfSensor}			[1]	{\SetOfInputLocations_{#1}}
\newcommand{\SetOfInputLocationsOfAuxiliaryGrid}	[0]	{\SetOfInputLocations^{G}}
%
\newcommand{\Measurement}									[0]	{y}
\newcommand{\MeasurementOfSensor}							[1]	{\Measurement_{#1}}
\newcommand{\MeasurementOfIndex}							[1]	{\Measurement^{#1}}
\newcommand{\MeasurementOfSensorAndIndex}					[2]	{\MeasurementOfSensor{#1}^{#2}}
\newcommand{\MeasurementOfAuxiliaryGridAndIndex}			[1]	{\MeasurementOfSensor{G}^{#1}}
\newcommand{\MeasurementOfAuxiliaryGridAndSensorAndIndex}	[2]	{\MeasurementOfSensor{G, #1}^{#2}}
%
\newcommand{\SetOfMeasurements}							[0]	{\mathcal{Y}}
\newcommand{\SetOfMeasurementsOfSensor}					[1]	{\SetOfMeasurements_{#1}}
\newcommand{\SetOfMeasurementsOfAuxiliaryGrid}			[0]	{\SetOfMeasurements^{G}}
\newcommand{\SetOfMeasurementsOfAuxiliaryGridOfSensor}	[1]	{\SetOfMeasurementsOfAuxiliaryGrid_{#1}}
\newcommand{\SetOfTransformedMeasurementsOfAuxiliaryGridOfSensor}	[1]	{\overline{\SetOfMeasurementsOfAuxiliaryGrid_{#1}}}
%
\newcommand{\MeasurementNoise}									[0]	{\nu}
\newcommand{\MeasurementNoiseOfSensor}							[1]	{\MeasurementNoise_{#1}}
\newcommand{\MeasurementNoiseOfIndex}							[1]	{\MeasurementNoise^{#1}}
\newcommand{\MeasurementNoiseOfSensorAndIndex}					[2]	{\MeasurementNoiseOfSensor{#1}^{#2}}
\newcommand{\MeasurementNoiseOfAuxiliaryGridAndIndex}			[1]	{\MeasurementNoiseOfSensor{G}^{#1}}
\newcommand{\MeasurementNoiseOfAuxiliaryGridAndSensorAndIndex}	[2]	{\MeasurementNoiseOfSensor{G, #1}^{#2}}
%
\newcommand{\SetOfMeasurementNoises}			[0]	{\mathcal{V}}
\newcommand{\SetOfMeasurementNoisesOfSensor}	[1]	{\SetOfMeasurementNoises_{#1}}
\newcommand{\SetOfMeasurementNoisesOfAuxiliaryGridOfSensor}	[1]	{\SetOfMeasurementNoises_{#1}^{G}}


\newcommand{\CovarianceOfMeasurementNoise}					[0]	{\sigma^{2}}
\newcommand{\CovarianceOfMeasurementNoiseOfSensor}			[1]	{\CovarianceOfMeasurementNoise_{#1}}
\newcommand{\CovarianceOfMeasurementNoiseOfSensorAndIndex}	[2]	{\CovarianceOfMeasurementNoise_{#1 , #2}}
\newcommand{\CovarianceOfMeasurementNoiseOfAuxiliaryGridAndSensorAndIndex}	[2]	{{\CovarianceOfMeasurementNoise_{#1 , #2}}^{G}}
\newcommand{\CovarianceOfSetOfMeasurementNoises}			[0]	{\Sigma_{\SetOfMeasurementNoises}}
\newcommand{\CovarianceOfSetOfMeasurementNoisesOfAuxiliaryGrid}	[0]	{\Sigma_{\SetOfMeasurementNoises}^{G}}
\newcommand{\CovarianceOfSetOfMeasurementNoisesOfSensor}		[1]	{\Sigma_{\SetOfMeasurementNoises, #1}}
\newcommand{\CovarianceOfSetOfMeasurementNoisesOfAuxiliaryGridOfSensor}		[1]	{\Sigma_{\SetOfMeasurementNoises, #1}^{G}}


\newcommand{\JitterOnInputLocation}						[0]	{u}
\newcommand{\JitterOnInputLocationOfSensorAndIndex}		[2]	{\JitterOnInputLocation_{#1}^{#2}}
%
\newcommand{\JitterOnInputLocationMaximalAmplitude}		[0]	{\ell}


\newcommand{\Probability}			[0]	{\mathbb{P}}
\newcommand{\ProbabilityOf}			[1]	{\Probability \left[ #1 \right]}
\newcommand{\ProbabilityOfGiven}	[2]	{\ProbabilityOf{ #1 \; \left| \; #2 \right.}}


\newcommand{\ProbabilityDistribution}		[0]	{P}
\newcommand{\ProbabilityDistributionOf}		[1]	{\ProbabilityDistribution \left( #1 \right)}
\newcommand{\ProbabilityDistributionOfGiven}[2]	{\ProbabilityDistributionOf{ #1 \left| #2 \right. }}
%
\newcommand{\ProbabilityDistributionOfRV}	[1]	{\ProbabilityDistribution_{#1}}
\newcommand{\ProbabilityDistributionOfRVOf}	[2]	{\ProbabilityDistributionOfRV{#1} \left( #2 \right)}
\newcommand{\ProbabilityDistributionOfRVOfGiven}	[3]
		   {\ProbabilityDistributionOfRVOf{#1}{ #2 \left| #3 \right. }}


\newcommand{\ProbabilityDensity}			[0]	{p}
\newcommand{\ProbabilityDensityOf}			[1]	{\ProbabilityDensity \left( #1 \right)}
\newcommand{\ProbabilityDensityOfGiven}		[2]	{\ProbabilityDensityOf{ #1 \left| #2 \right. }}
%
\newcommand{\ProbabilityDensityOfRV}		[1]	{\ProbabilityDensity_{#1}}
\newcommand{\ProbabilityDensityOfRVOf}		[2]	{\ProbabilityDensityOfRV{#1} \left( #2 \right)}
\newcommand{\ProbabilityDensityOfRVOfGiven}	[3] {\ProbabilityDensityOfRVOf{#1}{ #2 \left| #3 \right. }}


\newcommand{\ProbabilityMassFunction}		[0]	{m}
\newcommand{\ProbabilityMassFunctionOf}		[1]	{\ProbabilityMassFunction \left( #1 \right)}
\newcommand{\ProbabilityMassFunctionOfGiven}[2]	{\ProbabilityMassFunctionOf{ #1 \left| #2 \right. }}
%
\newcommand{\ProbabilityMassFunctionOfRV}	[1]	{\ProbabilityMassFunction_{#1}}
\newcommand{\ProbabilityMassFunctionOfRVOf}	[2]	{\ProbabilityMassFunctionOfRV{#1} \left( #2 \right)}
\newcommand{\ProbabilityMassFunctionOfRVOfGiven}		[3]
		   {\ProbabilityMassFunctionOfRVOf{#1}{ #2 \left| #3 \right. }}


\newcommand{\Expectation}					[0]	{\mathbb{E}}
\newcommand{\ExpectationOf}					[1]	{\Expectation \left[ #1 \right]}
\newcommand{\ExpectationOfOnMeasure}		[2]	{\Expectation_{#2} \left[ #1 \right]}
\newcommand{\ExpectationOfGiven}			[2]	{\ExpectationOf{ #1 \; \left| \; #2 \right. }}
\newcommand{\ExpectationOfOnMeasureGiven}	[3]	{\ExpectationOfOnMeasure{#1 \; \left| \; #3 \right.}{#2}}


\newcommand{\Variance}				[0]	{\mathrm{var}}
\newcommand{\VarianceOf}			[1]	{\Variance \left( #1 \right)}
\newcommand{\VarianceOfGiven}		[2]	{\VarianceOf{ #1 \; \left| \; #2 \right. }}


\newcommand{\Covariance}			[0]	{\mathrm{cov}}
\newcommand{\CovarianceOf}			[2]	{\Covariance \left( #1, #2 \right)}
\newcommand{\CovarianceOfGiven}		[3]	{\Covariance \left( #1, #2 \; \left| \; #3 \right. \right)}


\newcommand{\BayesEstimatorOfGiven}	[2]	{\widehat{\Expectation} \left[ #1 \; \left| \; #2 \right. \right] }


\newcommand{\IndicatorFunctionOf}	[1]	{\mathds{1} \left\lbrace #1 \right\rbrace}


\newcommand{\SufficientStatistic}	[0]	{T}
\newcommand{\SufficientStatisticOf}	[1]	{\SufficientStatistic \left( #1 \right)}

\newcommand	{\SuchThat}				{s.t.\ }

\newcommand	{\Section}				[0]	{Sec.}
\newcommand	{\Sections}				[0]	{Secc.}
\newcommand	{\Equation}				[0]	{Equ.}
\newcommand	{\Equations}			[0]	{Equu.}
\newcommand	{\Figure}				[0]	{Fig.}
\newcommand	{\Figures}				[0]	{Figg.}
\newcommand	{\Table}				[0]	{Tab.}
\newcommand	{\Tables}				[0]	{Tabb.}
\newcommand	{\Algorithm}			[0]	{Alg.}
\newcommand	{\Algorithms}			[0]	{Algg.}
\newcommand	{\Proposition}			[0]	{Prop.}
\newcommand	{\Propositions}			[0]	{Propp.}
\newcommand	{\Hypothesis}			[0]	{Hyp.}
\newcommand	{\Hypotheses}			[0]	{Hypp.}

\newcommand{\NetworkGraph}	[0]	{\mathcal{G}}
\newcommand{\SetOfNodes}	[0]	{\mathcal{N}}
\newcommand{\SetOfEdges}	[0]	{\mathcal{E}}




\newcommand{\InsertImage}[3] % path / height / width
{
	\begin{tikzpicture}[remember picture, overlay]
	\node
	[
		shape			= rectangle,		% shape
		minimum height	= #2cm,				% | minimum size of the node
		minimum width	= #3cm,				% |
 		path picture	=
		{\node at (path picture bounding box.center)
		{\includegraphics[height = #2cm, width = #3cm]
		{#1}};}
	]{};
	\end{tikzpicture}
}

\newcommand{\InsertImageAt}[5] % path / height / width / xshift / yshift
{
	\begin{tikzpicture}[remember picture, overlay]
	\node
	[
		shape			= rectangle,		% shape
		minimum height	= #2cm,				% | minimum size of the node
		minimum width	= #3cm,				% |
		xshift 			= #4cm,
		yshift			= #5cm,
 		path picture	=
		{\node at (path picture bounding box.center)
		{\includegraphics[height = #2cm, width = #3cm]
		{#1}};}
	]
	at (current page.center)
	{};
	\end{tikzpicture}
}
